%==============================================================================
% Tento soubor použijte jako základ
% This file should be used as a base for the thesis
% Autoři / Authors: 2008 Michal Bidlo, 2022 Jaroslav Dytrych
% Kontakt pro dotazy a připomínky: sablona@fit.vutbr.cz
% Contact for questions and comments: sablona@fit.vutbr.cz
%==============================================================================
% kódování: UTF-8 (zmena prikazem iconv, recode nebo cstocs)
% encoding: UTF-8 (you can change it by command iconv, recode or cstocs)
%------------------------------------------------------------------------------
% zpracování / processing: make, make pdf, make clean
%==============================================================================
% Soubory, které je nutné upravit nebo smazat: / Files which have to be edited or deleted:
%   doc-20-bibliography.bib - literatura / bibliography
%   doc-01-chapters.tex - obsah práce / the thesis content
%   doc-01-chapters-en.tex - obsah práce v angličtině / the thesis content in English
%   doc-30-appendices.tex - přílohy / appendices
%   doc-30-appendices-en.tex - přílohy v angličtině / appendices in English
%==============================================================================
% \documentclass[]{fitthesis} % bez zadání - pro začátek práce, aby nebyl problém s překladem
% \documentclass[english]{fitthesis} % without assignment - for the work start to avoid compilation problem
% \documentclass[zadani]{fitthesis} % odevzdani do IS VUT a/nebo tisk s barevnými odkazy - odkazy jsou barevné
\documentclass[english,zadani]{fitthesis} % for submission to the IS VUT and/or print with color links - links are color
%\documentclass[zadani,print]{fitthesis} % pro černobílý tisk - odkazy jsou černé
% \documentclass[english,zadani,print]{fitthesis} % for the black and white print - links are black
%\documentclass[zadani,cprint]{fitthesis} % pro barevný tisk - odkazy jsou černé, znak VUT barevný
%\documentclass[english,zadani,cprint]{fitthesis} % for the print - links are black, logo is color
% * Je-li práce psaná v anglickém jazyce, je zapotřebí u třídy použít 
%   parametr english následovně:
%   If thesis is written in English, it is necessary to use 
%   parameter english as follows:
%      \documentclass[english]{fitthesis}
% * Je-li práce psaná ve slovenském jazyce, je zapotřebí u třídy použít 
%   parametr slovak následovně:
%   If the work is written in the Slovak language, it is necessary 
%   to use parameter slovak as follows:
%      \documentclass[slovak]{fitthesis}
% * Je-li práce psaná v anglickém jazyce se slovenským abstraktem apod., 
%   je zapotřebí u třídy použít parametry english a enslovak následovně:
%   If the work is written in English with the Slovak abstract, etc., 
%   it is necessary to use parameters english and enslovak as follows:
%      \documentclass[english,enslovak]{fitthesis}

% Základní balíčky jsou dole v souboru šablony fitthesis.cls
% Basic packages are at the bottom of template file fitthesis.cls
% zde můžeme vložit vlastní balíčky / you can place own packages here


% Pro seznam zkratek lze využít balíček Glossaries - nutno odkomentovat i níže a při kompilaci z konzoly i v Makefile (plnou verzi pro Perl, nebo lite)
% The Glossaries package can be used for the list of abbreviations - it is necessary to uncomment also below. When compiling from the console also in the Makefile (full version for Perl or lite)
%\usepackage{glossaries}
%\usepackage{glossary-superragged}
%\makeglossaries 

% Nastavení cesty k obrázkům
% Setting of a path to the pictures
\graphicspath{{figures/}{./figures/}}
%\graphicspath{{obrazky-figures/}{../obrazky-figures/}}

%---rm---------------
\renewcommand{\rmdefault}{lmr}%zavede Latin Modern Roman jako rm / set Latin Modern Roman as rm
%---sf---------------
\renewcommand{\sfdefault}{qhv}%zavede TeX Gyre Heros jako sf
%---tt------------
\renewcommand{\ttdefault}{lmtt}% zavede Latin Modern tt jako tt

% vypne funkci šablony, která automaticky nahrazuje uvozovky,
% aby nebyly prováděny nevhodné náhrady v popisech API apod.
% disables function of the template which replaces quotation marks
% to avoid unnecessary replacements in the API descriptions etc.
\csdoublequotesoff

\usepackage{url}
\usepackage{multirow}
\usepackage{subcaption}
\usepackage{framed}
\usepackage{appendix}
\usepackage{booktabs}
% TODO remove?
% \usepackage{changepage}

\usepackage{siunitx}
\sisetup{
  group-minimum-digits=1,
}

% To fit the table of contents...
\usepackage{tocloft}
\setlength{\cftaftertoctitleskip}{18pt}
\setlength{\cftbeforetoctitleskip}{0pt}
\setlength{\cftbeforechapskip}{6pt}
% \renewcommand{\cftsecfont}{\small}
% \setlength{\cftbeforesecskip}{1pt}
% \setlength{\cftbeforesubsecskip}{6pt}
% \setlength{\cftbeforesubsubsecskip}{6pt}
% \renewcommand{\cfttoctitlefont}{\Large}
% \renewcommand{\cftsubsecfont}{\normalsize}
% \renewcommand{\cftsubsubsecfont}{\small}


\usepackage{threeparttable}


% =======================================================================
% balíček "hyperref" vytváří klikací odkazy v pdf, pokud tedy použijeme pdflatex
% problém je, že balíček hyperref musí být uveden jako poslední, takže nemůže
% být v šabloně
% "hyperref" package create clickable links in pdf if you are using pdflatex.
% Problem is that this package have to be introduced as the last one so it 
% can not be placed in the template file.
\ifWis
\ifx\pdfoutput\undefined % nejedeme pod pdflatexem / we are not using pdflatex
\else
  \usepackage{color}
  \usepackage[unicode,colorlinks,hyperindex,plainpages=false,pdftex]{hyperref}
  \definecolor{hrcolor-ref}{RGB}{223,52,30}
  \definecolor{hrcolor-cite}{HTML}{2F8F00}
  \definecolor{hrcolor-urls}{HTML}{092EAB}
  \hypersetup{
	linkcolor=hrcolor-ref,
	citecolor=hrcolor-cite,
	filecolor=magenta,
	urlcolor=hrcolor-urls
  }
  \def\pdfBorderAttrs{/Border [0 0 0] }  % bez okrajů kolem odkazů / without margins around links
  \pdfcompresslevel=9
\fi
\else % pro tisk budou odkazy, na které se dá klikat, černé / for the print clickable links will be black
\ifx\pdfoutput\undefined % nejedeme pod pdflatexem / we are not using pdflatex
\else
  \usepackage{color}
  \usepackage[unicode,colorlinks,hyperindex,plainpages=false,pdftex,urlcolor=black,linkcolor=black,citecolor=black]{hyperref}
  \definecolor{links}{rgb}{0,0,0}
  \definecolor{anchors}{rgb}{0,0,0}
  \def\AnchorColor{anchors}
  \def\LinkColor{links}
  \def\pdfBorderAttrs{/Border [0 0 0] } % bez okrajů kolem odkazů / without margins around links
  \pdfcompresslevel=9
\fi
\fi
% Řešení problému, kdy klikací odkazy na obrázky vedou za obrázek
% This solves the problems with links which leads after the picture
\usepackage[all]{hypcap}

% Hyperref tedro:
\addto\extrasenglish{
  \def\chapterautorefname{Chapter}
  \def\sectionautorefname{Section}
  \def\subsectionautorefname{Subsection}
  \def\subsubsectionautorefname{Subsubsection}
  \def\paragraphautorefname{Paragraph}
  \def\subparagraphautorefname{Subparagraph}
}

% TODO (nie tu) vloženie zadania (čítaj "moje záverečné práce" v studise)

% Informace o práci/projektu / Information about the thesis
%---------------------------------------------------------------------------
\projectinfo{
  %Prace / Thesis
  project={BP},            %typ práce BP/SP/DP/DR  / thesis type (SP = term project)
  year={2023},             % rok odevzdání / year of submission
  date=\today,             % datum odevzdání / submission date
  %Nazev prace / thesis title TODO treba česky? A je to dobre preložené?
  title.cs={Detekce a klasifikace vozidel pro vestavěné platformy},  % název práce v češtině či slovenštině (dle zadání) / thesis title in czech language (according to assignment)
  title.en={Detection and Classification of Vehicles for Embedded Platforms}, % název práce v angličtině / thesis title in english
  %title.length={14.5cm}, % nastavení délky bloku s titulkem pro úpravu zalomení řádku (lze definovat zde nebo níže) / setting the length of a block with a thesis title for adjusting a line break (can be defined here or below)
  %sectitle.length={14.5cm}, % nastavení délky bloku s druhým titulkem pro úpravu zalomení řádku (lze definovat zde nebo níže) / setting the length of a block with a second thesis title for adjusting a line break (can be defined here or below)
  %dectitle.length={14.5cm}, % nastavení délky bloku s titulkem nad prohlášením pro úpravu zalomení řádku (lze definovat zde nebo níže) / setting the length of a block with a thesis title above declaration for adjusting a line break (can be defined here or below)
  %Autor / Author
  author.name={Patrik},   % jméno autora / author name
  author.surname={Skaloš},   % příjmení autora / author surname 
  % author.title.p={}, % titul před jménem (nepovinné) / title before the name (optional)
  % author.title.a={}, % titul za jménem (nepovinné) / title after the name (optional)
  %Ustav / Department
  department={UPGM}, % doplňte příslušnou zkratku dle ústavu na zadání: UPSY/UIFS/UITS/UPGM / fill in appropriate abbreviation of the department according to assignment: UPSY/UIFS/UITS/UPGM
  % Školitel / supervisor
  supervisor.name={Jakub},   % jméno školitele / supervisor name 
  supervisor.surname={Špaňhel},   % příjmení školitele / supervisor surname
  supervisor.title.p={Ing.},   %titul před jménem (nepovinné) / title before the name (optional)
  % supervisor.title.a={},    %titul za jménem (nepovinné) / title after the name (optional)
  % keywords anglicky TODO
  keywords.en={vehicle detection, traffic surveillance, object detection,
  convolutional neural networks, real-time, trade-offs, embedded devices,
  low-performance devices, YOLO, YOLOv8, NVIDIA Jetson, Raspberry Pi, OpenMMLab,
  MMYOLO, quantization, ONNX Runtime, TensorRT},
  % keywords slovensky TODO
  keywords.cs={detekcia vozidiel, monitorovanie dopravy, detekcia objektov,
  konvolučné neurónové siete, detekcia v reálnom čase, kompromisy, vstavané
  zariadenia, nízkovýkonné zariadenia, YOLO, YOLOv8, NVIDIA Jetson, Raspberry
  Pi, OpenMMLab, MMYOLO, kvantizácia, ONNX Runtime, TensorRT},
  % Abstrakt anglicky TODO
  abstract.en={
    This paper evaluates the performance trade-offs of state-of-the-art YOLOv8
    object detectors for vehicle detection in surveillance-type images on
    embedded and low-performance devices. YOLOv8 models of varying sizes,
    including one with the lightweight MobileNetV2 backbone and YOLOv8-femto
    with fewer than \num{60000} parameters, were benchmarked across six devices,
    including three NVIDIA Jetson embedded platforms and the low-performance
    Raspberry Pi 4B. Various factors influencing performance were considered,
    such as weight quantization, input resolutions, inference backends, and
    batch sizes during inference. This study provides valuable insights into the
    development and deployment of vehicle detectors on a diverse range of
    devices, including low-performance CPUs and specialized embedded platforms.
  },
  abstract.cs={
    Táto práca hodnotí kompromisy rýchlosti a presnosti najmodernejších
    detektorov objektov YOLOv8 pre detekciu vozidiel v snímkoch z monitorovacích
    kamier na vstatných a nízkovýkonných zariadeniach. Modely YOLOv8 rôznych
    veľkostí, vrátane jedného s efektívnou sieťou MobileNetV2 na extrakciu
    príznakov a modelu YOLOv8-femto s menej ako \num{60000} parametrami, boli
    testované na šiestich zariadeniach, vrátane troch vstavaných platforiem z
    rodiny NVIDIA Jetson a počítačom Raspberry Pi 4B s nízkou výpočtovou silou.
    V práci boli zohľadnené rôzne faktory ovplyvňujúce výkonnosť modelov, ako
    napríklad ich kvantizácia, rozlíšenia vstupu, inferenčné knižnice a veľkosti
    dávok počas inferencie. Táto štúdia poskytuje užitočné informácie k vývoju a
    nasadeniu detektorov vozidiel na širokú škálu zariadení, od nízkovýkonných
    procesorov po špecializované vstavané platformy.
  },
  % Prohlášení TODO u anglicky psané práce anglicky
  declaration={
    I hereby declare that this Bachelor's thesis was prepared as an
    original work by the author under the supervision of Ing. Jakub Špaňhel. I
    have listed all the literary sources, publications and other sources, which
    were used during the preparation of this thesis.
  },
  % Poděkování TODO (nepovinné, nejlépe v jazyce práce; nechcete-li, zakomentujte pro skrytí nadpisu) / Acknowledgement (optional, ideally in the language of the thesis; comment out for hiding including heading)
  acknowledgment={I would like to express my gratitude to my supervisor, Ing.
  Jakub Špaňhel, for his valuable advice throughout the course of this
  research. His guidance and expertise helped me overcome difficult challenges
  and enabled me to focus on the important aspects of this work to successfully
  complete this thesis.
  },
  faculty={FIT} % FIT/FEKT/FSI/FA/FCH/FP/FAST/FAVU/USI/DEF
  % TODO REMOVE faculty.cs={Fakulta informačních technologií}, % Fakulta v češtině - pro využití této položky výše zvolte fakultu DEF / Faculty in Czech - for use of this entry select DEF above
  % TODO REMOVE faculty.en={Faculty of Information Technology}, % Fakulta v angličtině - pro využití této položky výše zvolte fakultu DEF / Faculty in English - for use of this entry select DEF above
  % TODO REMOVE department.cs={Ústav počítačové grafiky a multimédií}, % Ústav v češtině - pro využití této položky výše zvolte ústav DEF nebo jej zakomentujte / Department in Czech - for use of this entry select DEF above or comment it out
  % TODO REMOVE department.en={Department of Computer Graphics and Multimedia} % Ústav v angličtině - pro využití této položky výše zvolte ústav DEF nebo jej zakomentujte / Department in English - for use of this entry select DEF above or comment it out
}

% Rozšířený abstrakt (cca 3 normostrany) - lze definovat zde nebo výše /
% Extended abstract (approximately 3 standard pages) - can be defined here or
% above
% \extendedabstract{Do tohoto odstavce bude zapsán výtah (abstrakt) práce v českém (slovenském) jazyce.}
% Začít rozšířený abstrakt na liché stránce? / Should extended abstract start on the odd page?
%\extabstractodd{true}

% nastavení délky bloku s titulkem pro úpravu zalomení řádku - lze definovat zde nebo výše / setting the length of a block with a thesis title for adjusting a line break - can be defined here or above
\titlelength{14.5cm}
% nastavení délky bloku s druhým titulkem pro úpravu zalomení řádku - lze definovat zde nebo výše / setting the length of a block with a second thesis title for adjusting a line break - can be defined here or above
%\sectitlelength{14.5cm}
% nastavení délky bloku s titulkem nad prohlášením pro úpravu zalomení řádku - lze definovat zde nebo výše / setting the length of a block with a thesis title above declaration for adjusting a line break - can be defined here or above
%\dectitlelength{14.5cm}

% řeší první/poslední řádek odstavce na předchozí/následující stránce
% solves first/last row of the paragraph on the previous/next page
\clubpenalty=10000
\widowpenalty=10000

% checklist
\newlist{checklist}{itemize}{1}
\setlist[checklist]{label=$\square$}

% Kompilace po částech (rychlejší, ale v náhledu nemusí být vše aktuální)
% Compilation piecewise (faster, but not all parts in preview will be up-to-date)
% Další informace viz / For more information see https://www.overleaf.com/learn/latex/Multi-file_LaTeX_projects
% \usepackage{subfiles}

% Nechcete-li, aby se u oboustranného tisku roztahovaly mezery pro zaplnění stránky, odkomentujte následující řádek / If you do not want enlarged spacing for filling of the pages in case of duplex printing, uncomment the following line
% \raggedbottom

\begin{document}
  % Vysazeni titulnich stran / Typesetting of the title pages
  % ----------------------------------------------
  \maketitle
  % Obsah
  % ----------------------------------------------
  \setlength{\parskip}{0pt}

  {\hypersetup{hidelinks}\tableofcontents}
  
  % Seznam obrazku a tabulek (pokud prace obsahuje velke mnozstvi obrazku, tak se to hodi)
  % List of figures and list of tables (if the thesis contains a lot of pictures, it is good)
  \ifczech
    \renewcommand\listfigurename{Seznam obrázků}
  \fi
  \ifslovak
    \renewcommand\listfigurename{Zoznam obrázkov}
  \fi
  % {\hypersetup{hidelinks}\listoffigures}
  
  \ifczech
    \renewcommand\listtablename{Seznam tabulek}
  \fi
  \ifslovak
    \renewcommand\listtablename{Zoznam tabuliek}
  \fi
  % {\hypersetup{hidelinks}\listoftables}

  % Seznam zkratek / List of abbreviations
  %\ifczech
  %  \renewcommand*\glossaryname{Seznam zkratek}%
  %  \renewcommand*\entryname{Zkratka}
  %  \renewcommand*\descriptionname{Význam}
  %\fi
  %\ifslovak
  %  \renewcommand*\glossaryname{Zoznam skratiek}%
  %  \renewcommand*\entryname{Skratka}
  %  \renewcommand*\descriptionname{Význam}
  %\fi
  %\ifenglish
  %  \renewcommand*\glossaryname{List of abbreviations}%
  %  \renewcommand*\entryname{Abbreviation}
  %  \renewcommand*\descriptionname{Meaning}
  %\fi
  % Definice zkratek - z textu se odkazují např. \Gls{TF–IDF}
  % Definition of abbreviations - referred from the text e.g. \Gls{TF–IDF}
  %\newglossaryentry{TF–IDF}
  %{
  %  name={TF–IDF},
  %  description={Term Frequency-Inverse Document Frequency}
  %}
  % 
  %\setglossarystyle{superragged}
  %\printglossaries


  \ifODSAZ
    \setlength{\parskip}{0.5\bigskipamount}
  \else
    \setlength{\parskip}{0pt}
  \fi

  % vynechani stranky v oboustrannem rezimu
  % Skip the page in the two-sided mode
  \iftwoside
    \cleardoublepage
  \fi

  % Text prace / Thesis text
  % ----------------------------------------------
  % For compilation piecewise (see projekt.tex), it is necessary to uncomment it and change
% \documentclass[../projekt.tex]{subfiles}
% \begin{document}


% - Research pre viac aplikácií - napr. RPI s mini sieťou na riadenie križovatky
%   a congestion reduction, lebo tam netreba veľkú presnosť, a napr. jetson na
%   detekciu wrong-way a takých, a možno napr. colar na niečo medzi 

\newpage

\chapter{Introduction}
% TODO malo by to asi byť viac o tom, čo budem v papieri písať

% \todo{Na čo vehicle detection}
Vehicle detection in real-time is crucial for enhancing traffic safety and flow.
It can be used to manage traffic at an intersection using traffic lights, spot
speeding or wrong-way drivers, enhance route planning to ease congestion, alert
drivers to potentially dangerous situations, and offer insightful information
about driving behavior.

% \todo{ako na vehicle detection}
% TODO toto nejak skrátiť. Alebo to vôbec nepatrí do úvodu

There are various methods for detecting vehicles, each of which has
its advantages and limitations. In this paper, we focus on detection using a
camera-based traffic surveillance system. This approach was chosen because of
its low price, versatility, and ease of installation.

% \todo{detekcia s kamerami, výhody a nevýhody}

There are two main techniques for vehicle detection using a camera: image
processing and convolutional neural networks (CNNs). Both approaches are based
on analyzing images to identify patterns and features to detect vehicles. Image
processing techniques involve applying a series of pre-defined filters and
transformations to an image to extract relevant information. CNNs, on the other
hand, are a type of machine learning algorithm that learns to recognize patterns
and features through training. CNNs are generally more effective than image
processing techniques for vehicle detection, as they can learn to recognize
complex patterns and features that may be difficult to extract using pre-defined
filters and transformations, but they have traditionally required a significant
amount of computing power, commonly provided by a graphical processing
unit (GPU). However, recent developments suggest that it may now be possible to
perform CNN-based tasks in real time with reduced processing requirements. 
% Eliminating the need for a GPU and processing video on a CPU would greatly
% reduce cost and

% This will be evaluated in the current paper.

% CNNs typically require lots of processing
% power, but recent developments suggest that this might not be the case anymore.

% However, image processing can still be extremely
% helpful for simpler tasks or as a component of a larger vehicle detection
% system. Additionally, CNNs typically require a lot more processing power.

% \todo{cieľ a výsledky}
% The implementation of vehicle detection using CNNs requires the utilization of
% hardware with a high level of computational power, typically in the form of
% graphical processing units (GPUs). This hardware is usually located in a central
% server room and the video feed is transmitted over the internet, which can
% consume a significant amount of bandwidth.

% In this paper, we attempt to train a CNN with a smaller architecture that is
% capable of running on a single central processing unit (CPU). This would
% allow the hardware to be connected directly to the camera, with only the
% detection information being transmitted over the internet. This approach has the
% potential to reduce the computational demands on the hardware and potentially
% mitigate the need for high-powered GPUs or servers in a remote location.


In this paper, we attempt to train a CNN with a smaller architecture that is
able to run on a single central processing unit (CPU) instead of a GPU. This
would allow the system to process the camera feed on-site, rather than relying
on its transmission over the internet and remote processing. This would reduce
the cost of the system and the amount of network bandwidth required, as well as
expand the range of purposes for which the system can be utilized.


% \todo{štruktúra dokumentu (kde čo hľadať)}
% v sekcii X ukážeme, aké datasety sme použili a info o nich. v sekcii Y 
% aké modely sme použili a ako sme postupovali, neskôr experimenty pri ktorých
% sme dosiahli 999999% úspešnosť a tak...


\chapter{Background and Related Work}



\section{Common Approaches to Vehicle Detection}

This chapter contains a summary of commonly used solutions for the problem of
vehicle detection. First, we introduce systems that are not based on cameras,
and then we compare CNNs with image processing techniques. \todo{update}


\subsection*{Vehicle Detection Without a Camera}

\todo{Many different types of detectors exist but none is perfect so they're
used in harmony as complex systems consisting of many detectors?}

\subsubsection*{Magnetic Sensors}

One of the simplest solutions to detect vehicles is to use an induction
loop.\cite{MagneticSensors} The sensor, composed of a single wire, can be buried
under concrete and detect the presence of metal objects passing by. With a
controller, induction loops typically provide data about vehicle presence, but
using more advanced algorithms, speed, approximate classification and much more
can be determined from this simple sensor. Although the design is very simple,
installation is not and induction loops can even get damaged over time, while
repairs call for temporary shutdowns of roads. It is worth noting that there
are many other types of magnetic sensors, which can provide more detailed and
accurate data with simpler sensor installation, but the idea stays the same.

\subsubsection*{Ultrasonic Sensors}

Another type of inexpensive and simple detector is an ultrasonic
sensor.\cite{UltrasonicSensors} These sensors can operate in a variety of
conditions and are typically mounted on existing infrastructure. These distance
measurement sensors are usually used to detect the presence and distance of nearby
vehicles and their speed, but they can be utilized in many different ways to
even provide a simple shape of a vehicle to classify it.

\subsubsection*{Radars and Lidars}

Radars (Radio Detection and Ranging), which can also be mounted on existing
infrastructure above ground, work similarly to ultrasonic sensors, but a single
radar can oversee a much wider area of a road.\cite{RadarSensors} They are
usually used to detect the presence, speed, heading and shape of a vehicle. The
shape can then be used to predict a class of the vehicle. Lidars (Light
Detection and Ranging) can be used in a similar way, but are far more accurate,
which makes them better at detecting shapes and locations of objects.



\subsection*{Vehicle Detection Based on Image Processing}

% Kamery sú fajn, univerzálne, neinvazívne

% Ako to funguje?

% Treba stabilizáciu, homography (projective transform), detekciu, ktorá tiež veľa stojí
% K tomu, je ťažké to prispôsobiť na sneh, dážď, tornádo, divné vozidlá a tak
% A konfigurácia závisí od kamery a na inú kameru by to trebalo prispôsobiť

% However, they often have a problem with illumination changes, rain or snow,
% shadows, occlusions or noise. 

% Of course, there are methods for eliminating most
% of these problems, most importantly shadow removal\cite{ShadowRemoval}, but
% these methods add to the computational requirements of the system. 

% As our research suggests, the image processing approach is highly suitable for
% simpler tasks or lower accuracy requirements, but lacks versatility and is very
% hard to implement.

Image processing techniques take a camera feed as input and in addition to
detecting the presence, speed, heading and shape of a vehicle, they can
also provide its color, license plate and countless other characteristics,
limited mostly by the software. These camera-based systems, which are
relatively inexpensive, can be mounted on existing infrastructure, do not emit
energy and are highly versatile, while providing a large amount of data.

An algorithm applies a series of pre-defined filters and transformations to an
image to extract patterns and features that resemble vehicles. Although modern
convolutional neural networks (explained in the following \autoref{CNNs}) are generally
more effective at recognizing more complex patterns, image processing can still
be extremely helpful for simpler tasks or as a component of a larger vehicle
detection system.

A huge amount of research has been conducted on using image processing to solve
the problem of vehicle detection.\cite{ImageProcessingOverview} Many of the
techniques proposed can operate in real-time and are very reliable in typical
environmental conditions. However, they can be sensitive to changes in
illumination or have their performance affected by rain, snow, shadows,
occlusions, or noise. While there are methods for addressing some of these
challenges, such as shadow removal techniques\cite{ShadowRemoval}, they add to
the computational requirements of the system. Overall, our research suggests
that the image processing approach is well-suited for simpler tasks or systems
with lower accuracy requirements, but it is often difficult to implement and may
lack versatility in more complex or demanding scenarios.



% \section{Vehicle Detection With a Camera}

% Camera-based systems, which are relatively inexpensive, can be mounted on
% existing infrastructure, do not emit energy and are highly versatile, while
% providing a large amount of data. In addition to detecting the presence, speed,
% heading and shape of a vehicle, these systems can also provide its color,
% license plate and countless other characteristics.

% There are two main techniques for vehicle detection using a camera: image
% processing and convolutional neural networks (CNNs). Both approaches are based
% on analyzing images to identify patterns and features to detect vehicles. Image
% processing techniques involve applying a series of pre-defined filters and
% transformations to an image to extract relevant information. CNNs, on the other
% hand, are a type of machine learning algorithm that learns to recognize patterns
% and features through training. CNNs are generally more effective than image
% processing techniques for vehicle detection, as they can learn to recognize
% complex patterns and features that may be difficult to extract using pre-defined
% filters and transformations. However, image processing can still be extremely
% helpful for simpler tasks or as a component of a larger vehicle detection
% system. Additionally, CNNs typically require a lot more processing power.

% \subsection*{Image Processing Techniques}

% % Treba stabilizáciu, homography (projective transform), detekciu, ktorá tiež veľa stojí
% % K tomu, je ťažké to prispôsobiť na sneh, dážď, tornádo, divné vozidlá a tak
% % A konfigurácia závisí od kamery a na inú kameru by to trebalo prispôsobiť

% % However, they often have a problem with illumination changes, rain or snow,
% % shadows, occlusions or noise. 

% % Of course, there are methods for eliminating most
% % of these problems, most importantly shadow removal\cite{ShadowRemoval}, but
% % these methods add to the computational requirements of the system. 

% % As our research suggests, the image processing approach is highly suitable for
% % simpler tasks or lower accuracy requirements, but lacks versatility and is very
% % hard to implement.

% A huge amount of research has been conducted on using image processing to solve
% the problem of vehicle detection.\cite{ImageProcessingOverview} Many of the
% techniques proposed can operate in real-time and are very reliable in typical
% environmental conditions.  However, they can be sensitive to changes in
% illumination or have their performance affected by rain, snow, shadows,
% occlusions, or noise. While there are methods for addressing some of these
% challenges, such as shadow removal techniques\cite{ShadowRemoval}, they add to
% the computational requirements of the system. Overall, our research suggests
% that the image processing approach is well-suited for simpler tasks or systems
% with lower accuracy requirements, but it is often difficult to implement and may
% lack versatility in more complex or demanding scenarios.



% \subsection*{Convolutional Neural Networks}
% % rozdiel od image processing prístupu a proste čo robia. Nie ako fungujú

\section{Convolutional Neural Networks}
\label{CNNs}

\todo{ako fungujú?}

% Ako fungujú CNNs + pár príkladov? Asi to skôr v related work
% lightweight CNNs? Asi skôr v related work
%



\section{Object Detection?}


\section{Light-weight Detectors}


\section{Vehicle Detection}



\section{Tools and Libraries}


\subsection*{LabelBox annotation app?}


\subsection*{MMYOLO Library}

In this project, we utilize an open-source library
called MMYOLO\todo{cite} which is a part of the
OpenMMLab project\todo{cite?}. It is an extension of the MMDetection\todo{cite}
library, which provides a flexible framework for various object detection tasks,
supporting many state-of-the-art object detection models. The MMYOLO library
instead only focuses on object detectors from the YOLO family and at the time of
writing, supports the state-of-the-art YOLO detector\,--\,YOLOv8. Some of the
key features of the library include:
% Other notable features of the library include:

\begin{itemize}
    \item It is based on the PyTorch\footnote{PyTorch is a popular open-source
    machine learning framework used in computer vision and natural language
    processing} machine learning framework
    \item It builds upon and uses other open-source libraries from the OpenMMLab
    project, such as MMCV\footnote{MMCV is a library for computer vision
    research including building blocks for convolutional neural networks, tools
    for image processing, transformations and much more},
    MMEngine\footnote{MMEngine library serves as the training engine for all
    OpenMMLab codebases, supporting hundreds of algorithms frequently used in
    deep learning} and MMDeploy\footnote{MMDeploy library provides tools for
    deploying deep learning models}
    \item Library's modular design allows for easy customization and extension
    of the codebase
    \item It includes pre-trained models and their configurations which makes
    training and comparing different models fast and simple, while making it
    possible to use the transfer learning technique\todo{footnote ak som
    nevysvetlil transfer learning v teórii} which significantly speeds up the
    process.
    \item Implementation of YOLO-specific components, such as the CSPDarknet and
    PANet backbone networks
    \item Support for YOLO-specific training techniques, like data augmentations
    or loss functions
\end{itemize}

\todo{In practice - proste upraviť config a trénovať}




\chapter{Datasets}

Big and high-quality datasets are very important when training a CNN-based
detector. In this section, used datasets are listed and analyzed. First, the
criteria for recognizing an appropriate dataset for our task are explained. Each
chosen dataset is then analyzed individually. Finally, a summary of all used
datasets is provided.

\section{Dataset Criteria}

Here, we briefly explain the most important criteria for selecting datasets to
be used in a project like this.

\subsection*{Camera Angle}

Since we're building a detector for a camera mounted on infrastructure, it is
recommended to use datasets containing surveillance-type images. Datasets
containing only images taken from, for example, a car dashboard camera, were
therefore disregarded.

\subsection*{Classes}

If we don't want to re-label the dataset manually, its classes must be mappable
to \textit{our} classes. In this work, 8 object classes are considered:
\begin{itemize}
    \item Bicycle
    \item Motorcycle (any two-wheeled motorized vehicle)
    \item Passenger Car
    \item Transporter (or a van, pick-up truck, etc.)
    \item Bus (including a minibus)
    \item Truck
    \item Trailer
    \item Unknown
\end{itemize}

Many available datasets didn't annotate some of these classes or aggregated some of them into one
and were therefore ignored.

\subsection*{Diversity of Images}

For the trained detector to generalize well, it's important for a dataset to
contain images with different camera angles, lighting conditions, weather, etc.
60 FPS continuous video does not bring much of an advantage.

\subsection*{Dataset Quality}

We observed that many datasets contained incorrect annotations or
classifications. It is important to check the dataset and either fix faulty
annotations, ignore incorrectly annotated images or even disregard the whole
dataset.



\section{Individual Datasets}

This section individually analyzes all datasets used in this work.
\autoref{DatasetsSummary} compares these datasets on a higher level for an
overview. Example frames of each dataset are shown in figure \todo{examples -
na jednom obrázku example z každého datasetu} \todo{pod každý dataset dať ref na
example snímky?} \todo{ref na example snímky}.


\subsection*{UA-DETRAC}
\label{DetracDataset}

The DETRAC dataset \cite{detrac} provided by the University at Albany is the
biggest and the most important dataset for this work, originally containing
\num{1274055} annotations of \num{8250} vehicles in \num{138252} images. The
dataset is provided as frames from 100 video sequences of \SI{25}{fps} with the
resolution being $960 \times 540$ pixels. The length of these sequences varies, but
is usually between 30 seconds and 90 seconds. The video is of a surveillance
type and almost all sequences have a unique point of view, usually from a
bridge. Many sequences were recorded in rain or at night and there is no lens
flare from cars' headlights.

However, there are several issues with this dataset:
\begin{itemize}
    \item Bicycles, motorcycles and a few vehicles that cannot be classified are
    not annotated at all
    \item Bounding boxes are often loose and do not fit tightly to the objects
    \item Vehicles near the edge of the frame, although fully visible,
    sometimes lack labels
    \item In several sequences, vehicles are tracked and annotated even on
    frames on which they are fully occluded (by other vehicles or infrastructure)
    \item Vehicle annotations are inconsistent in relation to masks, as some are
    labeled even when masked, while others are left unlabeled even when already
    fully visible outside of a mask. This is likely due to the camera being
    hand-held while the mask is static throughout the sequence
\end{itemize}

These problems were fixed by importing the dataset to a labeling application
LabelBox \cite{labelbox}, adjusting the masks (while also making them dynamic),
annotating or masking bicycles, motorcycles and "unknown" vehicles and finally,
carefully repairing individual annotations if needed. Many sequences were hectic
or were annotated so poorly that reannotating would be too time-consuming, so
only 71 of 100 sequences were reannotated.

The reannotated dataset contains \num{733833} annotations in \num{99771}
images and these classes (the dataset contains no trailers):
\begin{itemize}
    \item Bicycle
    \item Motorcycle
    \item Car
    \item Bus
    \item Van - mapped to \textit{transporter}
    \item Others - mapped to \textit{truck}
    \item Unknown
\end{itemize}


\subsection*{Miovision Traffic Camera Dataset}

The MIO-TCD (Miovision traffic camera dataset) \cite{MIO2018} is another huge
and very important dataset. Images are taken at different times of day by
thousands of traffic cameras in Canada and the United States. Roughly $79\%$ of
all images are of resolution $720 \times 480$ pixels, but the image quality
seems to be lower. The rest is of resolution $342 \times 228$ pixels. \todo{ref
na example fotku}

The dataset consists of two parts: \textit{Classification dataset} and
\textit{Localization dataset}. Only the \textit{Train} subset of the
\textit{Classification} part is used here because the \textit{Test} subset is
not annotated.

The part used contains \num{351549} objects of these classes:
\begin{itemize}
    \item Pedestrian - ignored
    \item Bicycle
    \item Motorcycle
    \item Car
    \item Pickup truck - mapped to \textit{transporter}
    \item Work van - mapped to \textit{transporter}
    \item Bus
    \item Articulated truck - mapped to \textit{truck}bridge. Many sequences were recorded in rain or at night and there is no lens flare from
    cars' headlights.
    \item Single unit truck - mapped to \textit{truck}
    \item Non-motorized vehicle - mapped to \textit{trailer}
    \item Motorized vehicle - mapped to \textit{unknown}
\end{itemize}
The processed dataset (without pedestrian annotations) contains \num{344416}
objects in \num{110000} images.

The images are of low quality, there aren't many bicycles and motorcycles, and
more pictures with different weather and lighting conditions are needed.

\todo{example fotka 720x480?}


\subsection*{AAU RainSnow Traffic Surveillance Dataset}
\label{AAUDataset}

Another important dataset is the \textit{AAU RainSnow} dataset
\cite{Bahnsen2019}. The authors mounted two synchronized (one RGB and one
thermal) cameras on street lamps at seven different Danish intersections to take
5-minute long videos at different lighting and weather conditions - night and
day, rain and snow. They then extracted \num{2200} frames from the videos and
annotated them on a pixel-level. Several different types of masks were also
created and included in the dataset.

In our work, we only use the annotated frames from the RGB camera, containing
\num{13297} annotations in \num{2200} frames (before processing) of resolution
$640 \times 480$ pixels. \todo{ref na example fotku}

The dataset uses these 6 classes:
\begin{itemize}
    \item Pedestrian
    \item Bicycle
    \item Motorbike
    \item Car
    \item Bus
    \item Truck
\end{itemize}
This creates a problem - vehicles of our internal class \textit{transporter}
(van) don't have their own class in the dataset, but are classified as trucks.
However, the dataset is small and it is impossible to perfectly divide
transporters and trucks into two classes as there are many different models
between which a line cannot be drawn.  Ignoring this issue should therefore not
cause any problems.

Several other minor problems were found when processing this dataset:
\begin{itemize}
    \item Frames in groups \verb|Egensevej-1|, \verb|Egensevej-3| and
    \verb|Egensevej-5| are hardly usable because of the low-quality camera and
    challenging weather and lighting conditions, so they were dismissed
    \item Some frames had bounding boxes over the whole frame - this is most
    certainly an annotation error. These labels were ignored as well
    \item The mask for \verb|Hadsundvej| intersection didn't fully cover the
    area that should be ignored. This was fixed by simply editing the mask
\end{itemize}

After processing, the dataset contains \num{10545} objects in \num{1899} images.


\subsection*{Multi-View Traffic Intersection Dataset}

For the MTID dataset, the authors \cite{Jensen2020} recorded one intersection
from two points of view at \SI{30}{fps} - one camera was mounted on existing
infrastructure and one was attached to a hovering drone. The dataset contains
\num{65299} annotated objects in \num{5776} frames (equal share of frames for
both cameras).

\todo{ref na example fotku}

All annotated objects fall into one of four classes:
\begin{itemize}
    \item Bicycle
    \item Car
    \item Bus
    \item Lorry - mapped to \textit{truck}
\end{itemize}
This, at first, might not seem like enough, but a closer inspection of the
annotated frames reveals that there are no pedestrians or motorcycles. However,
similarly to the AAU dataset in \autoref{AAUDataset}, there are transporters in the frames
that are classified as trucks. Again, this issue is simply ignored.

When processing this dataset, two other problems were encountered:
\begin{itemize}
    \item Vehicles that are not on the road are not annotated, so they have to
    be masked out. This is not as easy for the drone video because the camera is
    moving, but it is still simple enough
    \item Many frames of the drone footage lack some or all labels and have to
    be ignored. Ranges of images numbers which are ignored: $[1,31]$,
    $[659,659]$, $[1001,1318]$ and $[3301,3327]$
\end{itemize}
The processed dataset contains \num{64979} objects in \num{5399} frames.


\subsection*{Night and Day Instance Segmented Park Dataset}

Another useful dataset is the \textit{NDISPark} \cite{Luca2022}, which contains
images of parked vehicles taken by a camera mounted on infrastructure. Although
the annotated part of the dataset only contains \num{142} frames after
processing, there are \num{3302} objects annotated in total. This still makes it
a tiny dataset, but it provides images of vehicles from many different points of
view and also contains many occluded vehicles. \todo{ref example fotku}
Additionally, all frames are \num{2400} px in width and within \num{908} px and
\num{1808} px in height.

This dataset does not contain any classifications, but luckily, it only contains
cars, \textit{transporters} and a few unattached car trailers. All
\textit{transporters} and \textit{trailers} were manually classified.


\subsection*{VisDrone Dataset}

The VisDrone dataset \cite{Zhu2022} is very different from all the previous
datasets since it doesn't just contain traffic surveillance images. Images are
taken by a camera mounted on a drone, from many different points of view. After
processing, there are \num{47720} annotated objects in \num{1610} frames.

This dataset might be a helpful addition to our datasets as it contains useful
negative images (many annotated people and new points of view) and it often
captures vehicles from a bid's-eye view.

Additionally, objects in this dataset are classified into 12 classes, which can
be easily mapped to our 8 \textit{internal} classes.


\section{Dataset processing}

\todo{v akom formáte sú datasety, do akého ich konvertujem, aké súbory vytváram a tak. Asi pokojne vcelku podrobne}

Before training, all datasets used in this project must be converted to a
unified format, object classes need to be mapped to be the same in each dataset,
some datasets include images with regions shat should be masked out and certain
subsets or images of some datasets need to be ignored.

For this, a python script was developed for each dataset. It first loads the
annotations from the original format - COCO \todo{ref?} for AAU RainSnow, MTID
and NDISPark, XML for DETRAC and different CSV formats for MIO-TCD and VisDrone.
The script then maps the classes to ones shown in \todo{ref} and if needed,
removes the ignored subsets or images before saving the labels in a COCO format.
The processing script for NDISPark dataset also corrects the object classes
before saving, as it was not done on the original ground-truth \todo{footnote
explain?} file (class corrections are defined in the script itself).

MMDetection's middle format was considered as a more efficient alternative to
the COCO format, but the COCO format is more popular and is supported by most
relevant application, while also being human-readable (MMDetection's middle
format is saved as a pickle \todo{footnote?} file), and most importantly, MMYOLO
and the newest version of MMDetection at the time of writing this paper (v3.0.0)
does not seem to fully support datasets in the middle format.

Several other python processing scripts were developed, to apply masks, combine
all ground-truth files into one and split the combined ground-truth file into
train, validation and test subsets. Additionally, scripts to review datasets
manually were created - one to visualize a dataset by simply adding bounding
boxes (with class labels) to the images and one converts the visualized images
to video (or videos).

A few more scripts were written, of which two are worth mentioning - one for
uploading the DETRAC dataset to LabelBox for reannotation and one for
downloading the reannotated labels.


\section{Summary of Datasets}

In \autoref{DatasetsSummary}, we compare used datasets on a higher level and
show the number of images and instances contained with additional comments. It
is clear that the DETRAC \autoref{DetracDataset} dataset amounts for most of our
data and might have been enough by itself, but to make this project as
successful as possible, every available useful dataset should be used. The
images from the selected datasets feature a great variety of lighting and
weather conditions, points of view, object scales and other relevant factors.
Additionally, in \autoref{DatasetsCounts} we show the number of annotated object
instances per class in all used datasets combined. \todo{Povedať niečo k class
inbalance?}


\begin{table}[t]
\centering
\label{DatasetsSummary}
\begin{tabular}{|l|rr|p{5cm}|}
    \hline
    Dataset tag & \# images & \# instances & Comments \\
    \hline
    DETRAC      &  \num{99771}  & \num{733833} & Large \newline Continuous video \newline High-quality camera \newline Different lighting conditions \\
    \hline
    MIO-TCD     &  \num{110000} & \num{344416} & Large \newline Low-quality images \\
    \hline
    AAU         &    \num{1899} &  \num{10545} & Small \newline Different weather conditions \\
    \hline
    MTID        &    \num{5399} &  \num{64979} & Small \newline Continuous video \\
    \hline
    NDISPark    &     \num{142} &   \num{3302} & Small \newline Occlusions \\
    \hline
    VisDrone    &    \num{1610} &  \num{47720} & Small \newline Negative images \newline New points of view \\
    \hline
    \hline
    Total       &  \num{218821} & \num{1204795} & - \\
    \hline
\end{tabular}
\caption{High-level comparison of used datasets}
\end{table}

\begin{table}[t]
\centering
\label{DatasetsCounts}
\begin{tabular}{|l|r|}
    \hline
    Class         & \# instances \\
    \hline
    Bicycle       &  \num{14036} \\
    Motorcycle    &  \num{17187} \\
    Passenger car & \num{916317} \\
    Transporter   & \num{123585} \\
    Bus           &  \num{64529} \\
    Truck         &  \num{38069} \\
    Trailer       &   \num{2360} \\
    Unknown       &  \num{28712} \\
    \hline
    \hline
    Total         & \num{1204795} \\
    \hline
\end{tabular}
\caption{Numbers of class instances in all datasets combined}
\end{table}


\todo{Treba potom niekde povedať o train/val/test split. Tu?}




\chapter{Model Configuration and Training}

In this chapter, an overview of configurations used to train the vehicle
detectors is displayed. Configuration of an object detection model contains a
set of parameters that define the model's behavior and the training, validation
and testing pipelines. These parameters can have a significant impact on the
model's speed and accuracy and tuning them is essential to achieve good results.

Most of the YOLOv8 parameters are left unchanged from the default configuration
of YOLOv8-m\footnote{TODO yolov8\_m\_syncbn\_fast\_8xb16-500e\_coco.py}, like: \\
\textbf{Optimizer:} Stochastic Gradient Descent with momentum \num{0.937} and weight decay \num{0.0005} \\
\textbf{Parameter scheduler:} Linear \verb|YOLOv5ParamScheduler| with learning rate factor of 0.01

To compensate for some datasets being smaller than others while being important
and of a high quality, a dataset wrapper \textit{RepeatDataset} is used, which
makes the underlying dataset n-times more frequent when training. The repetition
factor of each dataset is shown in \autoref{RepetitionFactors}.

\begin{table}[h]
\centering
\label{RepetitionFactors}
\small
\begin{tabular}{|c|c|c|c|}
    \hline
    Dataset name & \# images & Repetition factor & \# images after over-sampling \\
    \hline
    DETRAC       &  \num{99771} &  1 & \num{99771} \\
    MIO-TCD      & \num{110000} &  1 & \num{110000} \\
    AAU RainSnow &   \num{1899} &  3 & \num{5697} \\
    MTID         &   \num{5399} &  5 & \num{26995} \\
    NDISPark     &    \num{142} & 25 & \num{3550} \\
    VisDrone     &   \num{1610} &  4 & \num{6440} \\
    \hline
\end{tabular}
\caption{Repetition factors for each dataset used when training.}
\end{table}

Configuration entries that were adjusted individually for each model can be
found in \autoref{ModelSpecificConfigurations}.

\begin{table}[h]
\centering
\label{ModelSpecificConfigurations}
\small
\begin{tabular}{|p{3cm}|c|c|c|c|c|c|}
    \hline
    % \multicolumn{7}{|c|c|c|c|c|c|c|}{Model}
    \multirow{2}{*}{Model} & Deepen & Widen & \multirow{2}{*}{Learning rate} & \multirow{2}{*}{Batch size} & \multirow{2}{*}{Epochs} & Warmup \\
    & factor & factor & & & & epochs \\
    % & factor & factor & & & & epochs \\
    \hline
    YOLOv8 medium \newline $640 \times 384$ & & & & & & \\
    \hline
    YOLOv8 nano \newline $640 \times 384$   & & & & & & \\
    \hline
    YOLOv8 nano \newline $512 \times 288$   & & & & & & \\
    \hline
    YOLOv8 pico \newline $512 \times 288$   & & & & & & \\
    \hline
\end{tabular}
\caption{Training configurations for individual models.}
\end{table}


\section{Training Augmentation Pipeline}

Data augmentations are very important when training an object detector,
especially for detectors from the YOLO family. An augmentation in this context
refers to the process of applying various transformations to an image to
artificially increase the size and diversity of the training dataset, which
helps prevent overfitting and improves the generalization ability of the trained
model. To a convolutional neural network, even a tiny rotation, translation,
image flip, noise or color distortion makes an input image appear to be
something completely different, so applying these transformations randomly to
the input images is crucial to train a robust model.

\todo{nezabudnúť že posledných 10 epoch sa mení pipeline. Napísať to tu niekde}

% At the beginning of the pipeline, images and annotations are loaded.

\subsection*{Resize}

First, the image is resized to fit in the model's input size while, of course,
keeping the aspect ratio unchanged.

\subsection*{Pad}

\label{PadTransformation}
If the aspect ratio of the input image is not the same as the model's input, extra pixels around the input image must be added
to adapt to the model input's aspect ratio. The \textbf{color value} of the padded pixels is set to RGB(114, 114, 114).

\subsection*{Random Affine}

The \textit{YOLOv5RandomAffine} applies affine transformations to the image,
while randomly selecting the values from configured ranges. Parameters:\\
\textbf{Maximum translation ratio:} 0.05 \\
\textbf{Maximum rotation degree:} 5 \\
\textbf{Maximum shear degree:} 3 \\
\textbf{Scaling ratio} is set individually for each dataset as shown in \autoref{RandomAffineScalingRatios}.

\begin{table}[h]
\centering
\label{RandomAffineScalingRatios}
\begin{tabular}{|c|c|c|}
    \hline
    Dataset name & Minimum scaling ratio & Maximum scaling ratio \\
    \hline
    DETRAC       & 0.8 & 1.0 \\
    MIO-TCD      & 1.0 & 1.1 \\
    AAU RainSnow & 0.9 & 1.1 \\
    MTID         & 0.9 & 2.0 \\
    NDISPark     & 0.9 & 1.5 \\
    VisDrone     & 1.5 & 2.5 \\
    \hline
\end{tabular}
\caption{Image scaling ratios in \textit{RandomAffine} transformation for each dataset.}
\end{table}

The \textbf{color value for padding} around the tranformed image (if needed)
is set to RGB(114, 114, 114) to be the same as in the \textit{Pad} \ref{PadTransformation} transformation.

\subsection*{Cut Out}

The \textit{CutOut} transformations randomly selects regions of the image and fills them by a single color. Again,
parameters of this transformation are set individually for each dataset and are shown in \autoref{CutOutParams}.

\begin{table}[h]
\centering
\label{CutOutParams}
\begin{tabular}{|c|c|c|}
    \hline
    Dataset name & Number of regions & Size of a single region \\
    \hline
    DETRAC       &  6 & $22 \times 22$ px \\
    MIO-TCD      &  4 & $26 \times 26$ px \\
    AAU RainSnow &  8 & $10 \times 10$ px \\
    NDISPark     & 12 & $20 \times 20$ px \\
    MTID         & 12 & $10 \times 10$ px \\
    VisDrone     & 20 & $8 \times 8$ px \\
    \hline
\end{tabular}
\caption{\textit{CutOut} transformation parameters for each dataset.}
\end{table}

The \textbf{fill color value} is again set to RGB(114, 114, 114), same as for
padding in the previous transformations.

\subsection*{Custom Cut Out}

A custom cut out transformation was developed, similar to \textit{CutOut} used
in the previous step. Here, the regions are selected within each bounding box
with a certain probability, rather than being chosen randomly within the image.
Also, the region size is specified as a range of areas in relation to the
bounding box area - if the upper value is $10\%$ and a bounding box area is 100
pixels, the maximum region area can be 10 pixels.

With the boolean option \textit{random\_pixels} toggled, the color of each pixel
of a cut out region is generated randomly instead of filling it with a
pre-defined color. However, it was found to have no effect after testing.

The \textbf{probability} of a region being dropped from each bounding box is set
to 0.5 and the \textbf{region area} is set to be randomly selected from
interval $[5\%, 35\%]$.

\subsection*{Albumentations}

Albumentations\cite{Albumentations} are a popular open-source library for data
augmentation. The MMYOLO \todo{cite or something?} library provides the option
to use these transformations in the data augmentation pipeline. Settings are left unchanged from the original \todo{ref or something?}
\textit{YOLOv8-m} configuration: \\
\textbf{Blur probability:} 0.01 \\
\textbf{Median blur probability:} 0.01 \\
\textbf{Grayscale probability:} 0.01 \\
\textbf{CLAHE probability:} 0.01

\subsection*{HSV Random Augmentations}

The \textit{YOLOv5HSVRandomAug} simply adjusts the hue, saturation and value of
the image randomly.

\subsection*{Random Flip}

With \textbf{probability} of 0.5, the image is horizontally flipped using the
\textit{RandomFlip} augmentation.

\subsection*{Photometric Distortion}

The \textit{PhotoMetricDistortion} augmentation distorts an image sequentially,
while each transformation is applied with a probability of 0.5. It modifies the
brightness, contrast, converts color from BGR to HSV, modifies the saturation,
hue, converts from BGR to HSV, modifies the contrast and finally, randomly swaps
the color channels.

\subsection*{Filter Annotations}

As the last step in the pipeline, \textit{FilterAnnotations} is called to remove
bounding boxes with \textbf{width or height} lower than $8 \times 8$ pixels.






\chapter{Model Optimization}




\chapter{Inference and Deployment}




\chapter{Evaluation}







% Čo a ako som naprogramoval + zhodnotenie, porovnania a tak
\chapter{Experiments}

\chapter{Results}





\chapter{Future Work}





% Tu nemá byť nič nové - iba zhrnutie
\chapter{Conclusion}



% For compilation piecewise (see projekt.tex), it is necessary to uncomment it
% \end{document}
  
  % Kompilace po částech (viz výše, nutno odkomentovat a zakomentovat input výše)
  % Compilation piecewise (see above, it is necessary to uncomment it and comment out input above)
  %\subfile{chapters/projekt-01-uvod-introduction}
  % ...
  %\subfile{chapters/projekt-05-zaver-conclusion}

  % Pouzita literatura / Bibliography
  % ----------------------------------------------
\ifslovak
  \makeatletter
  \def\@openbib@code{\addcontentsline{toc}{chapter}{Literatúra}}
  \makeatother
  \bibliographystyle{bib-styles/Pysny/skplain}
\else
  \ifczech
    \makeatletter
    \def\@openbib@code{\addcontentsline{toc}{chapter}{Literatura}}
    \makeatother
    \bibliographystyle{bib-styles/Pysny/czplain}
  \else 
    \makeatletter
    \def\@openbib@code{\addcontentsline{toc}{chapter}{Bibliography}}
    \makeatother
    \bibliographystyle{bib-styles/Pysny/enplain}
  %  \bibliographystyle{alpha}
  \fi
\fi
  \begin{flushleft}
  \bibliography{doc-20-bibliography}
  \end{flushleft}

  % vynechani stranky v oboustrannem rezimu
  % Skip the page in the two-sided mode
  \iftwoside
    \cleardoublepage
  \fi

  % Prilohy / Appendices
  % ---------------------------------------------
  \appendix
\ifczech
  \renewcommand{\appendixpagename}{Přílohy}
  \renewcommand{\appendixtocname}{Přílohy}
  \renewcommand{\appendixname}{Příloha}
\fi
\ifslovak
  \renewcommand{\appendixpagename}{Prílohy}
  \renewcommand{\appendixtocname}{Prílohy}
  \renewcommand{\appendixname}{Príloha}
\fi
%  \appendixpage

% vynechani stranky v oboustrannem rezimu
% Skip the page in the two-sided mode
%\iftwoside
%  \cleardoublepage
%\fi
  
\ifslovak
%  \section*{Zoznam príloh}
%  \addcontentsline{toc}{section}{Zoznam príloh}
\else
  \ifczech
%    \section*{Seznam příloh}
%    \addcontentsline{toc}{section}{Seznam příloh}
  \else
%    \section*{List of Appendices}
%    \addcontentsline{toc}{section}{List of Appendices}
  \fi
\fi
  \startcontents[chapters]
  \setlength{\parskip}{0pt} 
  % seznam příloh / list of appendices
  
  \ifODSAZ
    \setlength{\parskip}{0.5\bigskipamount}
  \else
    \setlength{\parskip}{0pt}
  \fi
  
  % vynechani stranky v oboustrannem rezimu
  \iftwoside
    \cleardoublepage
  \fi
  
  % Přílohy / Appendices
  % % For compilation piecewise (see projekt.tex), it is necessary to uncomment it and change
%\documentclass[../projekt.tex]{subfiles}
%\begin{document}

% Placing of table of contents of the memory media here should be consulted with a supervisor
%\chapter{Contents of the included storage media}

%\chapter{Manual}

%\chapter{Configuration file}

%\chapter{Scheme of RelaxNG configuration file}

%\chapter{Poster}

% For compilation piecewise (see projekt.tex), it is necessary to uncomment it
%\end{document}
  
  % Kompilace po částech (viz výše, nutno odkomentovat)
  % Compilation piecewise (see above, it is necessary to uncomment it)
  %\subfile{doc-30-appendices}
  
\end{document}
